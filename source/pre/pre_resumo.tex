\chaves{ Teste de Software, Banco de Dados, Teste Funcional e Teste de Mutantes}

\begin{resumo} 
A garantia de qualidade de um software � importante para agrega��o de valor aos usu�rios que o utilizam. Uma das fases  para a obten��o dessa qualidade � a de Testes de Software. O teste de software pode ser aplicado em diferentes n�veis e utilizando-se de diferentes t�cnicas. A proposta para aumentar a confiabilidade do software � a aplica��o de  t�cnicas de testes, sejam elas funcional, estrutural ou baseado em erros. Muitas vezes as t�cnicas podem ser aplicadas de uma forma complementar, j� que n�o h� nenhuma t�cnica que � capaz de cobrir todos os aspectos do sistema. Neste Trabalho cont�m um estudo de caso de um software real, onde s�o aplicadas t�cnicas e crit�rios de testes funcional e teste de mutantes em uma Aplica��o de  Banco de Dados Relacional. S�o apresentados a contextualiza��o te�rica necess�ria para entender o estudo de caso, t�cnicas aplicadas e uma an�lise geral dos resultados obtidos. Este trabalho contribui com o entendimento e pr�tica das t�cnicas de testes em ABDR.
\end{resumo}


\documentclass[relatorio]{inf-ufg}
% Op��es da classe inf-ufg (ao usar mais de uma, separe por vi�rgulas)
%   [tese]         -> Tese de doutorado.
%   [dissertacao]  -> Disserta��o de mestrado (padr�o).
%   [monografia]   -> Monografia de especializa��o.
%   [relatorio]    -> Relat�rio final de gradua��o.
%   [abnt]         -> Usa o estilo "abnt-alf" de cita��o bibliogr�fica.
%   [nocolorlinks] -> Os links de navega��o no texto ficam na cor preta.
%                     Use esta op��o para gerar o arquivo para impress�o
%                     da vers�o final do seu texto!!!

\usepackage{placeins}
\usepackage{longtable}
\usepackage{seqsplit}
\usepackage{graphicx}


\begin{document}

\autor{Ana Let�cia Herculano da Silva}
\autorR{Silva, Ana Let�cia Herculano da}

\titulo{T�cnicas e Crit�rios de Testes em uma Aplica��o de Banco de Dados Relacioanal}
\subtitulo{Estudo de Caso}

\cidade{Goi�nia}
\dia{03} %
\mes{07} % Data da apresenta��o/defesa do trabalho
\ano{2017} % Formato num�rico: \dia{01}, \mes{01} e \ano{2009}

\orientador{Dr. C�ssio Leonardo Rodrigues}
\orientadorR{Dr. Rodrigues, C�ssio Leonardo}

\coorientador{Dr. Edmundo S�rgio Spoto}
\coorientadorR{Dr. Spoto, Edmundo S�rgio}

\universidade{Universidade Federal de Goi�s}
\uni{UFG}
\unidade{Instituto de Inform�tica}

\universidadeco{ Universidade Federal de Goi�s}
\unico{UFG}
\unidadeco{Instituto de Inform�tica}

\programa{Sistemas de Informa��o}
\concentracao{Teste de Software}

%-------------------------------------------------- ELEMENTOS PR�-TEXTUAIS %
\capa    % Gera o modelo da capa externa do trabalho
\publica % Gera a autoriza��o para publicacao em formato eletronico
\rosto   % Primeira folha interna do trabalho

\begin{aprovacao}
\banca{ Nome do membro da banca}{ Unidade acad�mica\ --  Sigla da universidade}
% Use o comando \profa se o membro da banca for do sexo feminino.
\profa{ Nome do membro da banca}{ Unidade acad�mica\ --  Sigla da universidade}
\end{aprovacao}
\direitos{Graduanda em Sistemas de Informa��o na UFG - Universidade Federal de Goi�s. Durante sua gradua��o, participou do projeto de homologa��o de PAF-ECF (Programa Aplicativo Fiscal - Emissor de Cupom Fiscal), prestando consultoria em homologa��o e teste de software para empresas de todo o Brasil. Possui certifica��o CTFL (Certified Tester, Foundation Level) e atuou em empresas goianas de software, especificamente na �rea de teste de software. Possui experi�ncia com sistemas de escritura��o fiscal, documentos fiscais eletr�nicos (NF-e, CT-e e afins), GRP (Government Resource Planning) e ITSM (IT Service Management). Atualmente trabalha como Analista de Testes em um sistema voltado para glosa de conv�nios hospitalares.}


\begin{dedicatoria}
 Dedicat�ria do trabalho a alguma pessoa, entidade, etc.
\end{dedicatoria}
\begin{agradecimentos}
Agrade�o primeiramente aos meus pais, Umbelina Luzia Herculano e Jos� Pereira da Silva, que incontestavelmente sempre me nutriram  de amor, carinho e aten��o para que eu pudesse alcan�ar meus objetivos. Tamb�m aos meus irm�os, Gustavo e J�lia Pereira Herculano, por me esperarem acordados, a chegar em casa a noite ap�s a aula para um abra�o e beijo de boa noite, os quais eu amo incondicionalmente. 

Agrade�o imensamente ao meu marido Danilo Guimar�es Justino Lemes pela for�a, paci�ncia, companheirismo e amor depositados para que eu tenha superado todos os desafios. Sem seu apoio eu n�o conseguiria.

Deixo minha gratid�o aos amigos que conquistei durante essa jornada: a Daniel Melo, que nunca mediu esfor�os para me ajudar, com sua sabedoria, paci�ncia e sensatez admir�vel. Ao Bruno Nogueira, sempre disposto a colaborar com seu conhecimento e sua incr�vel agilidade. A J�ssica Milene, que amenizou o sofrimento dessa jornada com seu carisma contagiante. Voc�s s�o hist�ria, que carrego comigo com bastante carinho. 

Agrade�o ao meu co-orientador Prof. Dr. Edmundo S�rgio Spoto, que com sua dedica��o, experi�ncia e conhecimento, soube me guiar com confian�a de bons resultados, e aprendizados para uma vida inteira. Sem sua instru��o n�o poderia ter alcan�ado este objetivo.
 
Agrade�o a Deus pela sabedoria a mim concedida e a saud�vel vida de minha av�, Erci Dias Herculano, que sempre zelou pelo meu bem.

\end{agradecimentos}



\epigrafe{Talvez n�o tenha conseguido fazer o melhor, mas lutei para que o melhor fosse feito. N�o sou o que deveria ser, mas Gra�as a Deus, n�o sou o que era antes.}
{Marthin Luther King Jr.}
{Pastor, ativista pol�tico, not�rio l�der do movimento dos direitos civis dos negros e Nobel da Paz de 1964}

\chaves{ Teste de Software, Banco de Dados, Teste Funcional e Teste de Mutantes}

\begin{resumo} 
A garantia de qualidade de um software � importante para agrega��o de valor aos clientes que o utilizam. Uma das fases  para a obten��o dessa qualidade � a de Testes de Software. O testes de software pode ser aplicado em diferentes n�veis e de diferentes t�cnicas. A proposta para aumentar a confiabilidade do software � aplicar t�cnicas de testes funcional ou baseado em erros ou estrutural com base nas depend�ncia de dados e integridade das informa��es trabalhadas no Banco de Dados. Muitas vezes todas as t�cnicas podem ser aplicadas de forma complementar, j� que nenhuma inclui a outra. Neste Trabalho foi aplicado um estudo de caso de um software real, onde � aplicado t�cnicas e crit�rios de testes funcional e teste de mutantes em uma Aplica��o de  Banco de Dados Relacional. S�o apresentados a contextualiza��o te�rica, um estudo de caso, os resultados obtidos e uma an�lise geral desses resultados.
\end{resumo}


\keys{Software Testing, Database, Fuctional Testing, Mutant Testing}

\begin{abstract}{\textless Work title\textgreater}
The book is on the table.
\end{abstract}


\tabelas[figtab]


%--------------------------------------------------------------- CAPITULOS %

\chapter{Introdu��o}
\label{cap:intro}

A onipresen�a de software ao redor do globo � indiscut�vel. Consumimos (e produzimos) em uma escala imensur�vel pelos mais otimistas de tempos passados. Temos contato

A qualidade de software aos poucos vem ocupando seu espa�o no processo de desenvolvimento de um software. O testes de software em si, � a principal etapa para que a qualidade seja garantida. Foram realizados alguns trabalhos e pesquisa com base nas diversas t�cnicas de testes, importantes para a realiza��o deste trabalho em si. 

Souza \cite{Souza} realizou uma pesquisa a qual foi a base para alguns crit�rios de testes aplicados neste trabalho. Ela prop�s alguns crit�rios de testes em Banco de Dados Relacional baseados na especifica��o de requisitos atrav�s da UML. Para apoiar a aplica��o dos conjuntos de crit�rios propostos, foi desenvolvido uma abordagem denominada mapeamento Conceitual de Informa��o.


\chapter{Teste de Software}
\label{cap:teste}

%% - - - - - - - - - - - - - - - - - - - - - - - - - - - - - - - - - - -
\section{Conceitos de Teste de Software}
\label{sec:conceitosTesteSoftware}


As op��es da classe s�o \verb|[tese]| (para tese de doutorado), \verb|[dissertacao]| (para disserta��o de mestrado), \verb|[monografia]| (para monografia de curso de especializa��o e \verb|[relatorio]| (para relat�rio final de curso de gradua��o). Se nenhuma op��o for declarada, o documento � considerado como uma disserta��o de mestrado. Se a op��o \verb|[abnt]| for utilizada, as cita��es bibliogr�ficas ser�o geradas conforme definido pelo grupo de trabalho \textsf{abnt-tex}. Contudo, o mais recomend�vel � n�o utilizar essa op��o. Com a op��o \verb|[nocolorlinks]| todos os {\em links} de navega��o no texto ficam na cor preta. O ideal � usar esta op��o para gerar o arquivo para impress�o, pois a qualidade da impress�o dos {\em links} fica superior.


%% - - - - - - - - - - - - - - - - - - - - - - - - - - - - - - - - - - -
\section{T�cnicas e Crit�rios de Teste de Software}
\label{sec:tecnicasCriteriosTesteSoftware}


%% - - - - - - - - - - - - - - - - - - - - - - - - - - - - - - - - - - -
\section{Teste Funcional}
\label{sec:testeFuncional}
Os elementos pr�--textuais s�o definidos p�gina por p�gina, conforme descritos a seguir:


%% lista!
\begin{itemize}
\item O primeiro argumento � o Perfil do aluno; e
\item O segundo argumento � a lista das palavras--chaves para a Ficha Catalogr�fica.
\end{itemize}


%% - - - - - - - - - - - - - - - - - - - - - - - - - - - - - - - - - - -
\section{Teste de Muta��o}
\label{sec:testeMutacao}

X-Man


%% - - - - - - - - - - - - - - - - - - - - - - - - - - - - - - - - - - -
\section{Banco de Dados}
\label{sec:bancoDeDados}
\chapter{Banco de Dados}
\label{cap:bancoDeDados}

% - - - - - - - - - - - - - - - - - - - - - - - - - - - - - - - - - - -
\section{Restri��es de Banco de Dados}
\label{sec:restricoesBd} 

\subsection{Restri��es de Dom�nio}
\label{restricoesDominio}

\subsection{Restri��es de Integridade Sem�ntica}
\label{restricoesIntegridadeSemantica}

\subsection{Restri��es de Integridade Referencial e de Chave Prim�ria}
\label{restricoesIntegridadeReferencial}

% - - - - - - - - - - - - - - - - - - - - - - - - - - - - - - - - - - -
\section{Aplica��es de Banco de Dados}
\label{sec:aplicacoesBd} 

% - - - - - - - - - - - - - - - - - - - - - - - - - - - - - - - - - - -
\section{Linguagem de Banco de Dados}
\label{sec:linguagemBd} 






\chapter{T�cnicas de Testes aplicadas em Banco de Dados Relacional}
\label{tecnicasTesteBD}

\section{Crit�rios de Teste Funcional em Aplica��es de Banco de Dados}
\label{testeFuncionalBd}

\subsection{Crit�rios  Para Testes Intra - Tabelas }
\label{criteriosIntraTabela}

\subsection{Crit�rios  Para Testes Inter-Tabelas }
\label{criteriosInterTabelas}

\section{T�cnica de Teste Mutante em Banco de Dados}
\label{testeMutanteBd}
\chapter{Estudo de Caso}
\label{estudoDeCaso}


\chapter{Plano de Teste}
\label{planoTeste}

% - - - - - - - - - - - - - - - - - - - - - - - - - - - - - - - - - - -
\section{Plano de Testes Baseado no Teste Funcional em BDR}
\label{sec:planoTestesBDR}

Segundo (Souza, 2008 apud Vilela, 2007\cite{Souza}), os Elementos Requeridos (ERs), s�o requisitos que satisfazem o teste, e eles s�o identificados atrav�s dos crit�rios de testes. 

Os elementos requeridos mostrados nas se��es a seguir s�o baseados nos crit�rios propostos na se��o \ref{sec:testeFuncionalBd}, sobre restri��es e cap�tulo 4, utilizando as t�cnicas de teste funcional, por parti��o, por classes de equival�ncia e an�lise do valor limite.

\subsection{Testes Funcionais Intra - Tabelas}
\label{sec:testesIntraTabela}

Na Tabela \ref{tab:elementoRequiridoIntraTabelaChavePrimariaRestricaoUnicidade} s�o apresentados os elementos requeridos pelo crit�rio Restri��o de Unicidade.  

\begin{table*}[p]
	\caption{Restri��o de Chave Prim�ria : Restri��o de Unicidade}
	\label{tab:elementoRequiridoIntraTabelaChavePrimariaRestricaoUnicidade} 
	\begin{tabular}{|c|c|p{3cm}|p{3cm}|p{3cm}|}
		\hline \textbf{Tabela} & \textbf{Atributo} & \textbf{Classes v�lidas} & \textbf{Classes inv�lidas} & \textbf{Sa�da esperada}\\ 
		\hline empresa 
		 & id 
		 & Duas tuplas n�o podem ter o mesmo valor de chave prim�ria do atributo \textit{id}
		 & Duas tuplas possuem o mesmo valor de chave prim�ria 
		 & A chave prim�ria da tabela empresa � �nica, ent�o o BD respeitou a restri��o de unicidade\\ 
		 
		 \hline empresa 
		 & cnpj
		 & Duas tuplas n�o podem ter o mesmo valor para o atributo \textit{cnpj} 
		 & Duas tuplas possuem o mesmo valor para o atributo \textit{cnpj}
		 & O atributo cnpj da tabela empresa � �nico, ent�o o BD respeitou a restri��o de unicidade\\
		
		 \hline empresa 
		 & razao\_social
		 & Duas tuplas n�o podem ter o mesmo valor para o atributo \textit{razao\_social} 
		 & Duas tuplas possuem o mesmo valor para o atributo \textit{razao\_social}
		 & O atributo razao\_social da tabela empresa � �nico, ent�o o BD respeitou a restri��o de unicidade\\
		 
		 \hline filial 
		 & id 
		 & Duas tuplas n�o podem ter o mesmo valor de chave prim�ria do atributo \textit{id}
		 & Duas tuplas possuem o mesmo valor de chave prim�ria 
		 & A chave prim�ria da tabela filial � �nica, ent�o o BD respeitou a restri��o de unicidade\\ 
		 
		 \hline filial 
		 & cnpj
		 & Duas tuplas n�o podem ter o mesmo valor para o atributo \textit{cnpj} 
		 & Duas tuplas possuem o mesmo valor para o atributo \textit{cnpj}
		 & O atributo cnpj da tabela filial � �nico, ent�o o BD respeitou a restri��o de unicidade\\
		 
		 \hline conta 
		 & id 
		 & Duas tuplas n�o podem ter o mesmo valor de chave prim�ria do atributo \textit{id}
		 & Duas tuplas possuem o mesmo valor de chave prim�ria 
		 & A chave prim�ria da tabela conta � �nica, ent�o o BD respeitou a restri��o de unicidade\\ 
		 
		 \hline conta 
		 & cnpj
		 & Duas tuplas n�o podem ter o mesmo valor para o atributo \textit{cnpj} 
		 & Duas tuplas possuem o mesmo valor para o atributo \textit{cnpj}
		 & O atributo cnpj da tabela conta � �nico, ent�o o BD respeitou a restri��o de unicidade\\
		 
		 \hline conta 
		 & razao\_social
		 & Duas tuplas n�o podem ter o mesmo valor para o atributo \textit{razao\_social} 
		 & Duas tuplas possuem o mesmo valor para o atributo \textit{razao\_social}
		 & O atributo razao\_social da tabela conta � �nico, ent�o o BD respeitou a restri��o de unicidade\\
		 
		 \hline plano 
		 & id 
		 & Duas tuplas n�o podem ter o mesmo valor de chave prim�ria do atributo \textit{id}
		 & Duas tuplas possuem o mesmo valor de chave prim�ria 
		 & A chave prim�ria da tabela plano � �nica, ent�o o BD respeitou a restri��o de unicidade\\ 
		 
		 \hline plano 
		 & nome
		 & Duas tuplas n�o podem ter o mesmo valor para o atributo \textit{nome} 
		 & Duas tuplas possuem o mesmo valor para o atributo \textit{nome}
		 & O atributo nome da tabela plano � �nico, ent�o o BD respeitou a restri��o de unicidade\\
		 
		 \hline conta\_plano 
		 & id 
		 & Duas tuplas n�o podem ter o mesmo valor de chave prim�ria do atributo \textit{id}
		 & Duas tuplas possuem o mesmo valor de chave prim�ria 
		 & A chave prim�ria da tabela conta\_plano � �nica, ent�o o BD respeitou a restri��o de unicidade\\ 
		 
		 \hline conta\_plano 
		 & id\_conta, id\_plano
		 & Duas tuplas n�o podem ter o mesmo valor para os atributos \textit{id\_conta, id\_plano} 
		 & Duas tuplas possuem o mesmo valor para os atributos \textit{id\_conta, id\_plano}
		 & Os atributos id\_conta, id\_plano da tabela conta\_plano s�o �nicos, ent�o o BD respeitou a restri��o de unicidade\\
		 
		\hline 
	\end{tabular} 
\end{table*}

Na Tabela \ref{tab:elementoRequiridoIntraTabelaChavePrimariaRestricaoEntidade} s�o apresentados os elementos requeridos pelo crit�rio Restri��o de Entidade. Nesta tabela s�o mostradas as chaves prim�rias de cada Entidade usada no exemplo da Figura \ref{fig:figura3}.

\begin{table*}[hp]
	\caption{Restri��o de Chave Prim�ria : Restri��o de Entidade}
	\label{tab:elementoRequiridoIntraTabelaChavePrimariaRestricaoEntidade} 
	\begin{tabular}{|c|c|p{3cm}|p{3cm}|p{3cm}|}
		\hline \textbf{Tabela} & \textbf{Atributo} & \textbf{Classes v�lidas} & \textbf{Classes inv�lidas} & \textbf{Sa�da esperada}\\ 
		\hline empresa 
		& id 
		& N�o pode conter valores nulos na chave prim�ria do atributo \textit{id}
		& A chave prim�ria \textit{id} cont�m valor nulo.
		& A chave prim�ria da tabela \textit{empresa} n�o possui valor nulo, ent�o o BD respeitou a restri��o de unicidade\\ 
		
		\hline filial 
		& id 
		& N�o pode conter valores nulos na chave prim�ria do atributo \textit{id}
		& A chave prim�ria \textit{id} cont�m valor nulo.
		& A chave prim�ria da tabela \textit{filial} n�o possui valor nulo, ent�o o BD respeitou a restri��o de unicidade\\ 
		
		
		\hline conta 
		& id 
		& N�o pode conter valores nulos na chave prim�ria do atributo \textit{id}
		& A chave prim�ria \textit{id} cont�m valor nulo.
		& A chave prim�ria da tabela \textit{conta} n�o possui valor nulo, ent�o o BD respeitou a restri��o de unicidade\\  
		
		\hline plano 
		& id
		& N�o pode conter valores nulos na chave prim�ria do atributo \textit{id}
		& A chave prim�ria \textit{id} cont�m valor nulo.
		& A chave prim�ria da tabela \textit{plano} n�o possui valor nulo, ent�o o BD respeitou a restri��o de unicidade\\ 
		
		\hline conta\_plano
		& id
		& N�o pode conter valores nulos na chave prim�ria do atributo \textit{id}
		& A chave prim�ria \textit{id} cont�m valor nulo.
		& A chave prim�ria da tabela \textit{conta\_plano} n�o possui valor nulo, ent�o o BD respeitou a restri��o de unicidade\\ 
		
		\hline preco\_modulo
		& id
		& N�o pode conter valores nulos na chave prim�ria do atributo \textit{id}
		& A chave prim�ria \textit{id} cont�m valor nulo.
		& A chave prim�ria da tabela \textit{preco\_modulo} n�o possui valor nulo, ent�o o BD respeitou a restri��o de unicidade\\ 
		
		\hline matriz\_empresa\_modulo
		& id
		& N�o pode conter valores nulos na chave prim�ria do atributo \textit{id}
		& A chave prim�ria \textit{id} cont�m valor nulo.
		& A chave prim�ria da tabela \textit{matriz\_empresa\_modulo} n�o possui valor nulo, ent�o o BD respeitou a restri��o de unicidade\\ 
		
		\hline usuario
		& id
		& N�o pode conter valores nulos na chave prim�ria do atributo \textit{id}
		& A chave prim�ria \textit{id} cont�m valor nulo.
		& A chave prim�ria da tabela \textit{usuario} n�o possui valor nulo, ent�o o BD respeitou a restri��o de unicidade\\ 
		
		\hline grupo
		& id
		& N�o pode conter valores nulos na chave prim�ria do atributo \textit{id}
		& A chave prim�ria \textit{id} cont�m valor nulo.
		& A chave prim�ria da tabela \textit{grupo} n�o possui valor nulo, ent�o o BD respeitou a restri��o de unicidade\\ 
	
		\hline 
	\end{tabular} 
\end{table*}


Nas Figuras \ref{fig:restricaoDominioEmpresa} a \ref{fig:restricaoDominioGrupoPermissao} s�o apresentados os elementos requeridos pelo Dom�nio de Atributos das tabelas definidas no estudo de caso da Figura \ref{fig:figura3}

 
 %% figura 1
 \begin{figure}[H]
 	\centering
 	\includegraphics[width=0.9\textwidth]{./fig/caso-teste/dominio/empresa}
 	\caption{Restri��o de Dom�nio de Atributo da tabela \textit{empresa}}
 	\label{fig:restricaoDominioEmpresa}
 \end{figure}

%% figura 1
\begin{figure}[H]
	\centering
	\includegraphics[width=0.9\textwidth]{./fig/caso-teste/dominio/filial}
	\caption{Restri��o de Dom�nio de Atributo da tabela \textit{filial}}
	\label{fig:restricaoDominioFilial}
\end{figure}

%% figura 1
\begin{figure}[H]
	\centering
	\includegraphics[width=0.9\textwidth]{./fig/caso-teste/dominio/conta}
	\caption{Restri��o de Dom�nio de Atributo da tabela \textit{conta}}
	\label{fig:restricaoDominioConta}
\end{figure}

%% figura 1
\begin{figure}[H]
	\centering
	\includegraphics[width=0.9\textwidth]{./fig/caso-teste/dominio/plano}
	\caption{Restri��o de Dom�nio de Atributo da tabela \textit{plano}}
	\label{fig:restricaoDominioPlano}
\end{figure}

%% figura 1
\begin{figure}[H]
	\centering
	\includegraphics[width=0.9\textwidth]{./fig/caso-teste/dominio/conta-plano}
	\caption{Restri��o de Dom�nio de Atributo da tabela \textit{conta-plano}}
	\label{fig:restricaoDominioContaPlano}
\end{figure}

%% figura 1
\begin{figure}[H]
	\centering
	\includegraphics[width=0.9\textwidth]{./fig/caso-teste/dominio/preco-modulo}
	\caption{Restri��o de Dom�nio de Atributo da tabela \textit{preco-modulo}}
	\label{fig:restricaoDominioPrecoModulo}
\end{figure}

%% figura 1
\begin{figure}[H]
	\centering
	\includegraphics[width=0.9\textwidth]{./fig/caso-teste/dominio/matriz-empresa-modulo}
	\caption{Restri��o de Dom�nio de Atributo da tabela \textit{matriz-empresa-modulo}}
	\label{fig:restricaoDominioMatrizEmpresaModulo}
\end{figure}

%% figura 1
\begin{figure}[H]
	\centering
	\includegraphics[width=0.9\textwidth]{./fig/caso-teste/dominio/usuario}
	\caption{Restri��o de Dom�nio de Atributo da tabela \textit{usuario}}
	\label{fig:restricaoDominioUsuario}
\end{figure}

%% figura 1
\begin{figure}[H]
	\centering
	\includegraphics[width=0.9\textwidth]{./fig/caso-teste/dominio/grupo}
	\caption{Restri��o de Dom�nio de Atributo da tabela \textit{grupo}}
	\label{fig:restricaoDominioGrupo}
\end{figure}

%% figura 1
\begin{figure}[H]
	\centering
	\includegraphics[width=0.9\textwidth]{./fig/caso-teste/dominio/grupo-permissao}
	\caption{Restri��o de Dom�nio de Atributo da tabela \textit{grupo-permissao}}
	\label{fig:restricaoDominioGrupoPermissao}
\end{figure}

%% figura 1
\begin{figure}[H]
	\centering
	\includegraphics[width=0.9\textwidth]{./fig/caso-teste/dominio/conta}
	\caption{Restri��o de Dom�nio de Atributo da tabela \textit{conta}}
	\label{fig:restricaoDominioConta}
\end{figure}

\subsection{Testes Funcionais Inter - Tabelas}
\label{sec:testesInterTabela}

Na Tabela de  \ref{fig:restricaoIntegridadeReferencialEmpresaFilialMatrizEmpresaModulo} a \ref{fig:restricaoIntegridadeReferencialUsuarioGrupoUsuarioGrupoGrupoPermissao} s�o apresentados os elementos requeridos pelo crit�rio Integridade Referencial de Relacionamentos, como especificado na Figura \ref{fig:figura3};

%% figura 1
\begin{figure}[H]
	\centering
	\includegraphics[width=0.9\textwidth]{./fig/caso-teste/integridade-referencial/empresa-filial-matrizEmpresaModulo}
	\caption{Restri��o de Integridade Referencial entre as tabelas \textit{empresa}, \textit{filial} e \textit{matriz\_empresa\_modulo}}
	\label{fig:restricaoIntegridadeReferencialEmpresaFilialMatrizEmpresaModulo}
\end{figure}

\begin{figure}[H]
	\centering
	\includegraphics[width=0.9\textwidth]{./fig/caso-teste/integridade-referencial/empresa-contaPlano-conta}
	\caption{Restri��o de Integridade Referencial entre as tabelas \textit{empresa}, \textit{conta\_plano} e \textit{conta}.}
	\label{fig:restricaoIntegridadeReferencialEmpresaContaPlanoConta}
\end{figure}


%% figura 1
\begin{figure}[H]
	\centering
	\includegraphics[width=0.9\textwidth]{./fig/caso-teste/integridade-referencial/contaPlano-conta-precoModulo}
	\caption{Restri��o de Integridade Referencial entre as tabelas \textit{conta\_plano}, \textit{plano} e \textit{preco\_modulo}}
	\label{fig:restricaoIntegridadeReferencialContaPlanoContaPrecoModulo}
\end{figure}


%% figura 1
\begin{figure}[H]
	\centering
	\includegraphics[width=0.9\textwidth]{./fig/caso-teste/integridade-referencial/usuario-grupo-usuarioGrupo-grupoPermissao}
	\caption{Restri��o de Integridade Referencial entre as tabelas \textit{usuario}, \textit{grupo}, \textit{usuario\_grupo} e \textit{grupo\_permisso}.}
	\label{fig:restricaoIntegridadeReferencialUsuarioGrupoUsuarioGrupoGrupoPermissao}
\end{figure}


\chapter{Resultados obtidos}
\label{cap:resultados}

Os casos de testes foram realizados no banco de dados Postgres atrav�s de comandos \nameref{sigla:DML}.
S�o apresentados  as entradas de dados, o teste aplicado, a sa�da esperada, a mensagem exibida pelo BD e o Resultado obtido pelo BD.

\section{Casos de Testes Intra-Tabelas}
\label{sec:resultadosCasosTestesIntraTabelas}

Ser�o apresentados os casos de testes Intra -Tabelas, distinguidos pelo tipo de crit�rio, e informando o :
\begin{itemize}
	\item N�mero de caso de Teste,
	\item A tabela referenciada,
	\item A entrada de dados,
	\item A sa�da esperada,
	\item A mensagem exibida pelo Banco de Dados ap�s a execu��o do testes e
	\item E o Resultado obtido pelo Banco de Dados
\end{itemize}

As Figuras \ref{fig:resultadoIntraTabelas1} a \ref{fig:resultadoIntraTabelasGrupoPermissao} s�o apresentados os resultados obtidos, conforme descritos na se��o \ref{sec:casosTestesIntraTabela}.

Os Scripts dos testes executados s�o apresentados no Ap�ndice \ref{apend:B}, ao final desse Trabalho.

\begin{figure}[H]
	\centering
	\includegraphics[width=1\textwidth]{./fig/resultados/intra-tabelas/unicidade/1}
	\caption{Resultados da execu��o dos testes Intra-tabelas 1}
	\label{fig:resultadoIntraTabelas1}
\end{figure}

\begin{figure}[H]
	\centering
	\includegraphics[width=1\textwidth]{./fig/resultados/intra-tabelas/unicidade/2}
	\caption{Resultados da execu��o dos testes Intra-tabelas 2}
	\label{fig:resultadoIntraTabelas2}
\end{figure}

\begin{figure}[H]
	\centering
	\includegraphics[width=1\textwidth]{./fig/resultados/intra-tabelas/unicidade/3}
	\caption{Resultados da execu��o dos testes Intra-tabelas 3}
	\label{fig:resultadoIntraTabelas3}
\end{figure}


\begin{figure}[H]
	\centering
	\includegraphics[width=1\textwidth]{./fig/resultados/intra-tabelas/dominio/empresa}
	\caption{Resultados da execu��o dos testes de dom�nio da tabela \textit{empresa}}
	\label{fig:resultadoIntraTabelasEmpresa}
\end{figure}

\begin{figure}[H]
	\centering
	\includegraphics[width=1\textwidth]{./fig/resultados/intra-tabelas/dominio/empresa-2}
	\caption{Resultados da execu��o dos testes de dom�nio da tabela \textit{empresa} (continua��o)}
	\label{fig:resultadoIntraTabelasEmpresaCont}
\end{figure}

\begin{figure}[H]
	\centering
	\includegraphics[width=1\textwidth]{./fig/resultados/intra-tabelas/dominio/filial}
	\caption{Resultados da execu��o dos testes de dom�nio da tabela \textit{filial}}
	\label{fig:resultadoIntraTabelasFilial}
\end{figure}

\begin{figure}[H]
	\centering
	\includegraphics[width=1\textwidth]{./fig/resultados/intra-tabelas/dominio/filial-2}
	\caption{Resultados da execu��o dos testes de dom�nio da tabela \textit{filial} (continua��o)}
	\label{fig:resultadoIntraTabelasFilialCont}
\end{figure}

\begin{figure}[H]
	\centering
	\includegraphics[width=1\textwidth]{./fig/resultados/intra-tabelas/dominio/conta}
	\caption{Resultados da execu��o dos testes de dom�nio da tabela \textit{conta}}
	\label{fig:resultadoIntraTabelasConta}
\end{figure}

\begin{figure}[H]
	\centering
	\includegraphics[width=1\textwidth]{./fig/resultados/intra-tabelas/dominio/conta-2}
	\caption{Resultados da execu��o dos testes de dom�nio da tabela \textit{conta} (continua��o)}
	\label{fig:resultadoIntraTabelasContaCont}
\end{figure}

\begin{figure}[H]
	\centering
	\includegraphics[width=1\textwidth]{./fig/resultados/intra-tabelas/dominio/plano}
	\caption{Resultados da execu��o dos testes de dom�nio da tabela \textit{plano}}
	\label{fig:resultadoIntraTabelasPlano}
\end{figure}

\begin{figure}[H]
	\centering
	\includegraphics[width=1\textwidth]{./fig/resultados/intra-tabelas/dominio/conta-plano}
	\caption{Resultados da execu��o dos testes de dom�nio da tabela \textit{conta\_plano}}
	\label{fig:resultadoIntraTabelasContaPlano}
\end{figure}

\begin{figure}[H]
	\centering
	\includegraphics[width=1\textwidth]{./fig/resultados/intra-tabelas/dominio/preco-modulo}
	\caption{Resultados da execu��o dos testes de dom�nio da tabela \textit{preco\_modulo}}
	\label{fig:resultadoIntraTabelasPrecoModulo}
\end{figure}

\begin{figure}[H]
	\centering
	\includegraphics[width=1\textwidth]{./fig/resultados/intra-tabelas/dominio/matriz-empresa-modulo}
	\caption{Resultados da execu��o dos testes de dom�nio da tabela \textit{matriz\_empresa\_modulo}}
	\label{fig:resultadoIntraTabelasMatrizEmpresaModulo}
\end{figure}

\begin{figure}[H]
	\centering
	\includegraphics[width=1\textwidth]{./fig/resultados/intra-tabelas/dominio/usuario}
	\caption{Resultados da execu��o dos testes de dom�nio da tabela \textit{usuario}}
	\label{fig:resultadoIntraTabelasUsuario}
\end{figure}

\begin{figure}[H]
	\centering
	\includegraphics[width=1\textwidth]{./fig/resultados/intra-tabelas/dominio/usuario-2}
	\caption{Resultados da execu��o dos testes de dom�nio da tabela \textit{usuario} (continua��o)}
	\label{fig:resultadoIntraTabelasUsuarioCont}
\end{figure}

\begin{figure}[H]
	\centering
	\includegraphics[width=1\textwidth]{./fig/resultados/intra-tabelas/dominio/grupo}
	\caption{Resultados da execu��o dos testes de dom�nio da tabela \textit{grupo}}
	\label{fig:resultadoIntraTabelasGrupo}
\end{figure}

\begin{figure}[H]
	\centering
	\includegraphics[width=1\textwidth]{./fig/resultados/intra-tabelas/dominio/grupo-permissao}
	\caption{Resultados da execu��o dos testes de dom�nio da tabela \textit{grupo\_permissao}}
	\label{fig:resultadoIntraTabelasGrupoPermissao}
\end{figure}


\section{Casos de Testes Inter-Tabelas}
\label{sec:resultadosCasosTestesInterTabelas}

As Figuras \ref{fig:resultadoInterTabelasChavePrimaria} e \ref{fig:resultadoInterTabelasAceitacao} apresentam os resultados obtidos.

\begin{figure}[H]
	\centering
	\includegraphics[width=1\textwidth]{./fig/resultados/inter-tabelas/1}
	\caption{Restri��o de Integridade Referencial Chave Prim�ria}
	\label{fig:resultadoInterTabelasChavePrimaria}
\end{figure}

\begin{figure}[H]
	\centering
	\includegraphics[width=1\textwidth]{./fig/resultados/inter-tabelas/2}
	\caption{Restri��o de Integridade Referencial -Aceita��o}
	\label{fig:resultadoInterTabelasAceitacao}
\end{figure}


\section{An�lise dos Resultados Obtidos na T�cnica Funcional}
\label{sec:analiseResultadoTecnicaFuncional}

De acordo com os testes realizados podemos observar que na t�cnica de Restri��o de unicidade, o banco se manteve �ntegro, todos os casos de testes foram executados com sucesso.  A integridade referencial foi mantida para todos os casos de testes executados e , definidos no Plano de testes.

J� na Integridade de dom�nio tivemos algumas falhas, onde o banco n�o teve o comportamento esperado, como n�o existir um check em cada vari�vel que tenha limites inferior e superior v�lidos dentro do escopo de cada tipo de dados por exemplos datas inv�lidas que n�o deveriam ser aceitas ou valores negativos quando a vari�vel deve ser somente positiva tipo idade, essas regras foram definidas no Plano de testes . Essas falhas s�o comuns, e pelo percentual de erros, pode-se afirmar que o banco de dados, est� 90\% de qualidade atingida, apenas 10\% foram defeitos ocorridos por esses motivos. 

Na Tabela \ref{tab:comparacao3Criterios} pode-se observar a rela��o entre os casos de testes executados e os erros encontrados, conforme descritos nas se��es \ref{sec:resultadosCasosTestesIntraTabelas} e \ref{sec:resultadosCasosTestesInterTabelas}:


\begin{table*}[hp]
	\centering
	\caption{Compara��o entre os 3 crit�rios utilizados}
	\label{tab:comparacao3Criterios} 
	\begin{tabular}{|c|c|c|c|}
		\hline \textbf{Crit�rio de Teste} & \textbf{Casos de Testes} & \textbf{Erros Encontrados} & \textbf{Porcentagem Eficaz} \\ 
		\hline \textbf{1 - Restri��o de Unicidade} & 28 & 0 & 100\%\\ 
		\hline \textbf{2 - Integridade de Dom�nio} & 112 & 9 & 91,96\%\\
		\hline \textbf{3 - Integridade Referencial} & 15 & 0 & 100\%\\
		\hline \textbf{Total} & \textbf{155} & \textbf{9} & \textbf{94,19\%}\\
		\hline 
	\end{tabular} 
\end{table*}


Nas Figuras \ref{subfig:unicidade}, \ref{subfig:dominio}, \ref{subfig:referencial} e \ref{subfig:total},  pode-se visualizar de uma melhor forma a porcentagem de casos de testes executados X os erros encontrados:

\begin{figure}[h]
	\centering
	%   \subfigure[][Primeira subfigura.]
	\subfigure[][Restri��o de Unicidade.]
	{
		\includegraphics[width=0.45\textwidth]{./fig/unicidade}
		\label{subfig:unicidade}
	} 
	\subfigure[Integridade de dom�nio.]
	{
		\includegraphics[width=0.45\textwidth]{./fig/dominio}
		\label{subfig:dominio}
	} 
	\subfigure[][Integridade Referencial.]
	{
		\includegraphics[width=0.45\textwidth]{./fig/referencial}
		\label{subfig:referencial}
	} 
	\subfigure[Total.]
	{
		\includegraphics[width=0.45\textwidth]{./fig/total}
		\label{subfig:total}
	}
	\caption{Restri��o de Unicidade {\subref{subfig:unicidade}}, Integridade de Dom�nio {\subref{subfig:dominio}}, Integridade Referencial{\subref{subfig:referencial}} e Total {\subref{subfig:total}}.}
	\label{fig:subfiguras1}
\end{figure}


\clearpage

\section{Aplica��o de Teste de Mutantes em Scripts de Banco de Dados (SQL)}
\label{sec:aplicacaoTesteMutantesEmScriptSql}

Nesta se��o ser� apresentado alguns scripts que ser�o aplicados os operadores de muta��o descritos na se��o T�cnicas de Teste Mutante em Banco de Dados \ref{sec:testeMutanteBd}. Os demais scripts est�o no ap�ndice \ref{apend:D}. Ser�o realizados 3 exemplos de scripts em \nameref{sigla:SQL} para criarmos os mutantes conforme os operadores de muta��o dados em Tuya et al \cite{Tuya2006}. Na Figura \ref{fig:operadores} mostra como os grupos de operadores de muta��o s�o selecionados e quais tipos s�o aplicados neste trabalho.

%% figura operadores
\begin{figure}[H]
	\centering
	\includegraphics[width=0.9\textwidth]{./fig/operadores}
	\caption{Defini��o de Operadores Utilizados}
	\label{fig:operadores}
\end{figure}

\pagebreak

\subsection{Caso 1}
\label{subsec:caso1}

No Caso 1, ser�o aplicados os subtipos \textbf{ROR} e \textbf{LCR} do grupo \textbf{OR}, pr� definidos na se��o T�cnicas de Teste de Mutante em Banco de Dados \ref{sec:testeMutanteBd}.

Query SQL: busca o plano e o usu�rio para cada filial de cada empresa.

\begin{verbatim}
SELECT E.id, E.razao_social, F.id, F.razao_social, P.nome, CP.id, CP.id_plano,
U.nome, C.id FROM filial F, empresa E, conta_plano CP, plano P, usuario U, 
conta C WHERE F.id_empresa = E.id and
E.id_plano = CP.id  and
P.id = CP.id_plano and
U.id_conta = C.id;
\end{verbatim}

Exemplos dos Mutantes  gerados no Caso 1:
 
\paragraph{M1.1}
\label{mutante:1.1}
\begin{Verbatim}
SELECT E.id, E.razao_social, F.id, F.razao_social, P.nome, CP.id, CP.id_plano,
U.nome, C.id FROM filial F, empresa E, conta_plano CP, plano P, usuario U, 
conta C WHERE F.id_empresa \textbf{\textcolor{red}{>}} E.id and
E.id_plano = CP.id  and
P.id = CP.id_plano and
U.id_conta = C.id;
\end{Verbatim}

\paragraph{M1.5}
\label{mutante:1.5}
\begin{Verbatim}
SELECT E.id, E.razao_social, F.id, F.razao_social, P.nome, CP.id, CP.id_plano,
U.nome, C.id FROM filial F, empresa E, conta_plano CP, plano P, usuario U, 
conta C WHERE F.id_empresa = E.id and
E.id_plano \textbf{\textcolor{red}{<}} CP.id  and
P.id = CP.id_plano and
U.id_conta = C.id;
\end{Verbatim}

\paragraph{M1.9}
\label{mutante:1.9}
\begin{Verbatim}
SELECT E.id, E.razao_social, F.id, F.razao_social, P.nome, CP.id, CP.id_plano,
U.nome, C.id FROM filial F, empresa E, conta_plano CP, plano P, usuario U, 
conta C WHERE 
F.id_empresa = E.id and
E.id_plano = CP.id  and
P.id \textbf{\textcolor{red}{<>}} CP.id_plano and
U.id_conta = C.id;
\end{Verbatim}

\paragraph{M1.13}
\label{mutante:1.13}
\begin{Verbatim}
SELECT E.id, E.razao_social, F.id, F.razao_social, P.nome, CP.id, CP.id_plano,
U.nome, C.id FROM filial F, empresa E, conta_plano CP, plano P, usuario U, 
conta C WHERE F.id_empresa = E.id \textbf{\textcolor{red}{or}}
E.id_plano = CP.id  and
P.id = CP.id_plano and
U.id_conta = C.id;
\end{Verbatim}

Na Tabela \ref{tab:execucaoMutantesCaso1} mostra os resultados da execu��o de cada mutante gerado no Caso 1 mostrando que n�o houve nenhum mutante vivo. Ou seja todos mutantes deram resultados diferenciados em rela��o ao Script Original. Desta forma os erros plantados na gera��o de cada mutante mostrado nos exemplos acima e colocados no Ap�ndice, mostraram que no script original n�o existem esses respectivos defeitos, o que � muito importante e d� seguran�a para a aplica��o.

% Tabela Execu��o dos Mutantes Caso1
\begin{table*}[h!p]
	\centering
	\caption{ Execu��o dos Mutantes \nameref{subsec:caso1}}
	\label{tab:execucaoMutantesCaso1} 
	\begin{tabular}{|c|c|}
		\hline \textbf{Script Referenciado} & \textbf{Resultado}\\ 
		\hline Original & 56 Linhas \\ 
		\hline \nameref{mutante:1.1} & 63 Linhas \\ 
		\hline M1.2 & 217 Linhas \\ 
		\hline M1.3 & 280 Linhas \\ 
		\hline M1.4 & 63 Linhas \\ 
		\hline \nameref{mutante:1.5} & 217 Linhas \\ 
		\hline M1.6 & 280 Linhas \\ 
		\hline M1.7 & 49 Linhas \\ 
		\hline M1.8 & 175 Linhas \\ 
		\hline \nameref{mutante:1.9} & 224 Linhas \\ 
		\hline M1.10 & 32 Linhas \\ 
		\hline M1.11 & 136 Linhas \\  
		\hline M1.12 & 168 Linhas \\ 
		\hline \nameref{mutante:1.13} & 19840 Linhas \\ 
		\hline M1.14 & 5160 Linhas \\ 
		\hline M1.15 & 10664 Linhas \\ 
		\hline 
	\end{tabular} 
\end{table*}

\pagebreak


\subsection{Caso 2}
\label{subsec:caso2}

No segundo exemplo,ser�o aplicados os subtipos \textbf{SUB} e \textbf{ROR}, do grupo \textbf{SC} e do grupo \textbf{OR}, respectivamente.

Query SQL: consulta o valor de mensal pago por cada usu�rio ativo em cada plano

\begin{verbatim}
SELECT u.nome, g.id, p.nome, pm.valor_mensal_fixo
FROM usuario U, grupo G, plano P, usuario_grupo UP, preco_modulo PM
WHERE U.id = UP.id_usuario and
G.id = UP.id_grupo and
PM.id_plano = P.id and
U.id in(SELECT distinct id from usuario WHERE ativo = true);
\end{verbatim}

Exemplos dos Mutantes  gerados no Caso 2: 

\paragraph{M2.1}
\label{mutante:2.1}
\begin{Verbatim}
SELECT u.nome, g.id, p.nome, pm.valor_mensal_fixo
FROM usuario U, grupo G, plano P, usuario_grupo UP, preco_modulo PM
WHERE U.id \textbf{\textcolor{red}{>}} UP.id_usuario and
G.id = UP.id_grupo and
PM.id_plano = P.id and
U.id in(SELECT distinct id from usuario WHERE ativo = true);
\end{Verbatim}

\paragraph{M2.9}
\label{mutante:2.9}
\begin{Verbatim}
SELECT u.nome, g.id, p.nome, pm.valor_mensal_fixo
FROM usuario U, grupo G, plano P, usuario_grupo UP, preco_modulo PM
WHERE U.id = UP.id_usuario and
G.id = UP.id_grupo and
PM.id_plano \textbf{\textcolor{red}{<>}} P.id and
U.id  in(SELECT distinct id from usuario WHERE ativo = true);
\end{Verbatim}

\paragraph{M2.10}
\label{mutante:2.10}
\begin{Verbatim}
SELECT u.nome, g.id, p.nome, pm.valor_mensal_fixo
FROM usuario U, grupo G, plano P, usuario_grupo UP, preco_modulo PM
WHERE U.id = UP.id_usuario and
G.id = UP.id_grupo and
PM.id_plano = P.id and
U.id \textbf{\textcolor{red}{not}} in(SELECT distinct id from usuario WHERE ativo = true);
\end{Verbatim}

Na Tabela \ref{tab:execucaoMutantesCaso2} mostra o resultado dos mutantes gerados nos mutantes gerados no Caso 2 mostrados anteriormente. Nesse caso tamb�m ilustra que os defeitos colocados em cada mutante n�o est� presente no Script original. No Caso 2 foram gerados 13 mutantes baseados nos operadores j� comentados anteriormente. Os Mutantes s�o apresentados no Ap�ndice \ref{apend:D}. 

% Tabela Execu��o dos Mutantes Caso2
\begin{table*}[hp]
	\centering
	\caption{ Execu��o dos Mutantes \nameref{subsec:caso2}}
	\label{tab:execucaoMutantesCaso2} 
	\begin{tabular}{|c|c|}
		\hline \textbf{Script Referenciado} & \textbf{Resultado}\\ 
		\hline Original & 288 Linhas \\ 
		\hline \nameref{mutante:2.1} & 3078 Linhas \\ 
		\hline M2.2 & 1254 Linhas \\ 
		\hline M2.3 & 4332 Linhas \\ 
		\hline M2.4 & 0 Linha \\ 
		\hline M2.5 & 152 Linhas \\ 
		\hline M2.6 & 1140 Linhas \\ 
		\hline M2.7 & 240 Linhas \\ 
		\hline M2.8 & 672 Linhas \\ 
		\hline \nameref{mutante:2.9} & 912 Linhas \\ 
		\hline \nameref{mutante:2.10} & 0 Linha \\ 
		\hline M2.11 & 0 Linha \\  
		\hline M2.12 & 0 Linha \\ 
		\hline M2.13 & 0 Linha \\ 
		\hline 
	\end{tabular} 
\end{table*}

\pagebreak

\subsection{Caso 3}
\label{subsec:caso3}

No Terceiro exemplo � utilizado o subtipo \textbf{LKE} tamb�m pertencente ao grupo \textbf{OR}; A quantidade desses mutantes � reduzida devido combina��o efetuada com 2 operadores, e a quantidade de dados no Banco.

Query SQL: busca todos usu�rios cujo nome comece com a letra  \textit{A} (mai�sculo)

\begin{verbatim}
SELECT * from Usuario
WHERE nome Like 'A%';
-- onde a letra A pode ser uma entrada de dado de um ou mais caracteres.
\end{verbatim}

Exemplos dos Mutantes  gerados no Caso 3: 

\paragraph{M3.1}
\label{mutante:3.1} 	 	 	 	
\begin{Verbatim}
SELECT * from Usuario
WHERE nome Like '\textbf{\textcolor{red}{A_}}';
\end{Verbatim}

\paragraph{M3.2}
\label{mutante:3.2} 
\begin{Verbatim}
SELECT * from Usuario
WHERE nome Like '\textbf{\textcolor{red}{a}}%';
\end{Verbatim}

\paragraph{M3.3}
\label{mutante:3.3}
\begin{Verbatim}
SELECT * from Usuario
WHERE nome Like '\textbf{\textcolor{red}{a_}}';
\end{Verbatim}

\paragraph{M3.4}
\label{mutante:3.4}
\begin{Verbatim}
SELECT * from Usuario
WHERE nome Like '\textbf{\textcolor{red}{C}}%';
\end{Verbatim}

\paragraph{M3.5}
\label{mutante:3.5} 
\begin{Verbatim}
SELECT * from Usuario
WHERE nome Like '\textbf{\textcolor{red}{C_}}';
\end{Verbatim}

\pagebreak

Na Tabela \ref{tab:execucaoMutantesCaso3} mostra o resultado dos mutantes gerados nos mutantes gerados no Caso 3 mostrados anteriormente. Nesse caso tamb�m ilustra que os defeitos colocados em cada mutante n�o est� presente no Script original. No Caso 3 foram gerados 5 mutantes baseados nos operadores j� comentados anteriormente. Os Mutantes s�o apresentados no Ap�ndice \ref{apend:D}. 

% Tabela Execu��o dos Mutantes Caso3
\begin{table*}[hp]
	\centering
	\caption{ Execu��o dos Mutantes \nameref{subsec:caso3}}
	\label{tab:execucaoMutantesCaso3} 
	\begin{tabular}{|c|c|}
		\hline \textbf{Script Referenciado} & \textbf{Resultado}\\ 
		\hline Original & 1 Linha \\ 
		\hline \nameref{mutante:3.1} & 0 Linha \\ 
		\hline \nameref{mutante:3.2} & 2 Linhas \\ 
		\hline \nameref{mutante:3.3} & 0 Linha \\ 
		\hline \nameref{mutante:3.4} & 6 Linhas \\ 
		\hline \nameref{mutante:3.5} & 0 Linha \\ 
		
		\hline 
	\end{tabular} 
\end{table*}

\section{An�lise dos Resultados obtidos em Testes de Mutante}
\label{sec:analiseResultadoTesteMutante}

Pode-se observar que o  teste de mutantes n�o tem efeito direto como a t�cnica de testes funcional possui no Banco de Dados (no projeto do Banco), j� o teste de mutante contribui  em verificar que os scripts de SQL usados nas aplica��es, est�o corretos, n�o possuindo nenhum defeito gerado pelos operadores de muta��o. Nos Mutantes gerados em nosso exemplo n�o ocorreram  defeitos nos scripts, ou seja, n�o houve nenhum mutante equivalente (quando o mutante gera a mesma sa�da do que o script original). 

Na Tabela \ref{tab:relacaoMutantesGeradosNosExemplos} � apresentada a rela��o de quantidade de mutantes gerados por cada grupo de cada caso (scripts) utilizado como exemplo para ilustrar essa pr�tica.


% Rela��o de Mutantes gerados nos exemplos
\begin{table*}[hp]
	\centering
	\caption{ Rela��o de Mutantes gerados nos exemplos}
	\label{tab:relacaoMutantesGeradosNosExemplos} 
	\begin{tabular}{|c|c|c|c|c|}
		\hline \textbf{Operador} & \textbf{\nameref{subsec:caso1}} & \textbf{\nameref{subsec:caso2}} & \textbf{\nameref{subsec:caso3}} & \textbf{Total}\\ 
		\hline \textbf{OR} & 15 & 9 & 5 & 29\\ 
		\hline \textbf{SC} & 0 & 1 & 0 & 1\\
		\hline  &  &  &  & 30\\
		
		\hline 
	\end{tabular} 
\end{table*}

\pagebreak

Na Figura \ref{fig:mutantes} � mostrado o total de mutantes em porcentagem de grupo.

%% figura mutantes
\begin{figure}[H]
	\centering
	\includegraphics[width=0.6\textwidth]{./fig/mutantes}
	\caption{ Porcentagem de Mutantes gerados por grupo}
	\label{fig:mutantes}
\end{figure}

\chapter{Conclusão e trabalhos futuros}
\label{conclusao}

%------------------------------------------------------------ BIBLIOGRAFIA %
\cleardoublepage
\nocite{*} %%% Retire esta linha para gerar a bibliografia com apenas as
           %%% refer�ncias usadas no seu texto!
\arial
\bibliography{./bib/modelo-tese} %%% Nomes dos seus arquivos .bib
\label{ref-bib}

%--------------------------------------------------------------- AP�NDICES %
\apendices

\chapter{Script de cria��o da base de dados}
\label{apend:A}
Ap�ndicess s�o iniciados com o comando \verb|\apendices|.
\chapter{Script de Inser��o inicial no Banco}
\label{apend:B}
Texto do Ap�ndice~\ref{apend:B}.

Ap�ndices s�o iniciados com o comando \verb|\apendices|.
\chapter{Scripts dos casos de testes}
\label{apend:C}
Texto do Apêndice~\ref{apend:C}.
\chapter{Scripts dos Testes de  Mutantes}
\label{apend:D}
Texto do Apêndice~\ref{apend:C}.


\end{document}
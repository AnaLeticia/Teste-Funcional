\chapter{Teste de Software}
\label{cap:teste}

%% - - - - - - - - - - - - - - - - - - - - - - - - - - - - - - - - - - -
\section{Conceitos de Teste de Software}
\label{sec:conceitosTesteSoftware}


As op��es da classe s�o \verb|[tese]| (para tese de doutorado), \verb|[dissertacao]| (para disserta��o de mestrado), \verb|[monografia]| (para monografia de curso de especializa��o e \verb|[relatorio]| (para relat�rio final de curso de gradua��o). Se nenhuma op��o for declarada, o documento � considerado como uma disserta��o de mestrado. Se a op��o \verb|[abnt]| for utilizada, as cita��es bibliogr�ficas ser�o geradas conforme definido pelo grupo de trabalho \textsf{abnt-tex}. Contudo, o mais recomend�vel � n�o utilizar essa op��o. Com a op��o \verb|[nocolorlinks]| todos os {\em links} de navega��o no texto ficam na cor preta. O ideal � usar esta op��o para gerar o arquivo para impress�o, pois a qualidade da impress�o dos {\em links} fica superior.


%% - - - - - - - - - - - - - - - - - - - - - - - - - - - - - - - - - - -
\section{T�cnicas e Crit�rios de Teste de Software}
\label{sec:tecnicasCriteriosTesteSoftware}


%% - - - - - - - - - - - - - - - - - - - - - - - - - - - - - - - - - - -
\section{Teste Funcional}
\label{sec:testeFuncional}
Os elementos pr�--textuais s�o definidos p�gina por p�gina, conforme descritos a seguir:


%% lista!
\begin{itemize}
\item O primeiro argumento � o Perfil do aluno; e
\item O segundo argumento � a lista das palavras--chaves para a Ficha Catalogr�fica.
\end{itemize}


%% - - - - - - - - - - - - - - - - - - - - - - - - - - - - - - - - - - -
\section{Teste de Muta��o}
\label{sec:testeMutacao}

X-Man


%% - - - - - - - - - - - - - - - - - - - - - - - - - - - - - - - - - - -
\section{Banco de Dados}
\label{sec:bancoDeDados}
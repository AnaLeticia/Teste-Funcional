
\chapter{Introdu��o}
\label{cap:intro}

A onipresen�a de software ao redor do globo � indiscut�vel. Consumimos (e produzimos) em uma escala imensur�vel pelos mais otimistas de tempos passados. Temos contato

A qualidade de software aos poucos vem ocupando seu espa�o no processo de desenvolvimento de um software. O testes de software em si, � a principal etapa para que a qualidade seja garantida. Foram realizados alguns trabalhos e pesquisa com base nas diversas t�cnicas de testes, importantes para a realiza��o deste trabalho em si. 

Souza \cite{Souza} realizou uma pesquisa a qual foi a base para alguns crit�rios de testes aplicados neste trabalho. Ela prop�s alguns crit�rios de testes em Banco de Dados Relacional baseados na especifica��o de requisitos atrav�s da UML. Para apoiar a aplica��o dos conjuntos de crit�rios propostos, foi desenvolvido uma abordagem denominada mapeamento Conceitual de Informa��o.


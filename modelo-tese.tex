% Exemplo de dissertacao do INF-UFG com texto em portugues formatado com LaTeX
\documentclass[relatorio]{inf-ufg}
% Op��es da classe inf-ufg (ao usar mais de uma, separe por vi�rgulas)
%   [tese]         -> Tese de doutorado.
%   [dissertacao]  -> Disserta��o de mestrado (padr�o).
%   [monografia]   -> Monografia de especializa��o.
%   [relatorio]    -> Relat�rio final de gradua��o.
%   [abnt]         -> Usa o estilo "abnt-alf" de cita��o bibliogr�fica.
%   [nocolorlinks] -> Os links de navega��o no texto ficam na cor preta.
%                     Use esta op��o para gerar o arquivo para impress�o
%                     da vers�o final do seu texto!!!

\usepackage{placeins}
%\usepackage[utf8]{inputenc}


%----------------------------------------------------- INICIO DO DOCUMENTO %
\begin{document}

%------------------------------------------ AUTOR, T�TULO E DATA DE DEFESA %
\autor{Ana Let\'icia Herculano da Silva}
\autorR{da Silva, Ana Let\'icia Herculano}

\titulo{T\'ecnicas e Crit\'erios de Testes em uma Aplica\c{c}\~ao de Banco de Dados Relacioanal}
\subtitulo{Estudo de Caso}

\cidade{Goi\^ania}
\dia{20} %
\mes{07} % Data da apresenta��o/defesa do trabalho
\ano{2017} % Formato num�rico: \dia{01}, \mes{01} e \ano{2009}

%-------------------------------------------------------------- ORIENTADOR %
\orientador{C\'assio Leonardo Rodrigues}
\orientadorR{Rodrigues, C\'assio Leonardo}


\coorientador{Dr. Edmundo S\'ergio Spoto}
\coorientadorR{Dr. Spoto, Edmundo S\'ergio}

%-------------------------------------------------- INSTITUI��O E PROGRAMA %
\universidade{Universidade Federal de Goi\'as}
\uni{UFG}
\unidade{Instituto de Inform\'atica}


\universidadeco{ Universidade Federal de Goi\'as}
\unico{UFG}
\unidadeco{Instituto de Inform\'atica}

\programa{Sistemas de Informa\c{c}\~ao}
\concentracao{Teste de Software}

%-------------------------------------------------- ELEMENTOS PR�-TEXTUAIS %
\capa    % Gera o modelo da capa externa do trabalho
\publica % Gera a autoriza��o para publicacao em formato eletronico
\rosto   % Primeira folha interna do trabalho

\begin{aprovacao}
\banca{ Nome do membro da banca}{ Unidade acad�mica\ --  Sigla da universidade}
% Use o comando \profa se o membro da banca for do sexo feminino.
\profa{ Nome do membro da banca}{ Unidade acad�mica\ --  Sigla da universidade}
\end{aprovacao}
\direitos{Graduanda em Sistemas de Informa��o na UFG - Universidade Federal de Goi�s. Durante sua gradua��o, participou do projeto de homologa��o de PAF-ECF (Programa Aplicativo Fiscal - Emissor de Cupom Fiscal), prestando consultoria em homologa��o e teste de software para empresas de todo o Brasil. Possui certifica��o CTFL (Certified Tester, Foundation Level) e atuou em empresas goianas de software, especificamente na �rea de teste de software. Possui experi�ncia com sistemas de escritura��o fiscal, documentos fiscais eletr�nicos (NF-e, CT-e e afins), GRP (Government Resource Planning) e ITSM (IT Service Management). Atualmente trabalha como Analista de Testes em um sistema voltado para glosa de conv�nios hospitalares.}


\begin{dedicatoria}
 Dedicat�ria do trabalho a alguma pessoa, entidade, etc.
\end{dedicatoria}
\begin{agradecimentos}
Agrade�o primeiramente aos meus pais, Umbelina Luzia Herculano e Jos� Pereira da Silva, que incontestavelmente sempre me nutriram  de amor, carinho e aten��o para que eu pudesse alcan�ar meus objetivos. Tamb�m aos meus irm�os, Gustavo e J�lia Pereira Herculano, por me esperarem acordados, a chegar em casa a noite ap�s a aula para um abra�o e beijo de boa noite, os quais eu amo incondicionalmente. 

Agrade�o imensamente ao meu marido Danilo Guimar�es Justino Lemes pela for�a, paci�ncia, companheirismo e amor depositados para que eu tenha superado todos os desafios. Sem seu apoio eu n�o conseguiria.

Deixo minha gratid�o aos amigos que conquistei durante essa jornada: a Daniel Melo, que nunca mediu esfor�os para me ajudar, com sua sabedoria, paci�ncia e sensatez admir�vel. Ao Bruno Nogueira, sempre disposto a colaborar com seu conhecimento e sua incr�vel agilidade. A J�ssica Milene, que amenizou o sofrimento dessa jornada com seu carisma contagiante. Voc�s s�o hist�ria, que carrego comigo com bastante carinho. 

Agrade�o ao meu co-orientador Prof. Dr. Edmundo S�rgio Spoto, que com sua dedica��o, experi�ncia e conhecimento, soube me guiar com confian�a de bons resultados, e aprendizados para uma vida inteira. Sem sua instru��o n�o poderia ter alcan�ado este objetivo.
 
Agrade�o a Deus pela sabedoria a mim concedida e a saud�vel vida de minha av�, Erci Dias Herculano, que sempre zelou pelo meu bem.

\end{agradecimentos}



\epigrafe{Talvez n�o tenha conseguido fazer o melhor, mas lutei para que o melhor fosse feito. N�o sou o que deveria ser, mas Gra�as a Deus, n�o sou o que era antes.}
{Marthin Luther King Jr.}
{Pastor, ativista pol�tico, not�rio l�der do movimento dos direitos civis dos negros e Nobel da Paz de 1964}

\chaves{ Teste de Software, Banco de Dados, Teste Funcional e Teste de Mutantes}

\begin{resumo} 
A garantia de qualidade de um software � importante para agrega��o de valor aos clientes que o utilizam. Uma das fases  para a obten��o dessa qualidade � a de Testes de Software. O testes de software pode ser aplicado em diferentes n�veis e de diferentes t�cnicas. A proposta para aumentar a confiabilidade do software � aplicar t�cnicas de testes funcional ou baseado em erros ou estrutural com base nas depend�ncia de dados e integridade das informa��es trabalhadas no Banco de Dados. Muitas vezes todas as t�cnicas podem ser aplicadas de forma complementar, j� que nenhuma inclui a outra. Neste Trabalho foi aplicado um estudo de caso de um software real, onde � aplicado t�cnicas e crit�rios de testes funcional e teste de mutantes em uma Aplica��o de  Banco de Dados Relacional. S�o apresentados a contextualiza��o te�rica, um estudo de caso, os resultados obtidos e uma an�lise geral desses resultados.
\end{resumo}


\keys{Software Testing, Database, Fuctional Testing, Mutant Testing}

\begin{abstract}{\textless Work title\textgreater}
The book is on the table.
\end{abstract}


\tabelas[figtabalgcod]
%Op��es:
%nada [] -> Gera apenas o sum�rio
%fig     -> Gera o sum�rio e a lista de figuras
%tab     -> Sum�rio e lista de tabelas
%alg     -> Sum�rio e lista de algoritmos
%cod     -> Sum�rio e lista de c�digos de programas
%
% Pode-se usar qualquer combina��o dessas op��es.
% Por exemplo:
%  figtab       -> Sum�rio e listas de figuras e tabelas
%  figtabcod    -> Sum�rio e listas de figuras, tabelas e
%                  c�digos de programas
%  figtabalg    -> Sum�rio e listas de figuras, tabelas e algoritmos
%  figtabalgcod -> Sum�rio e listas de figuras, tabelas, algoritmos e
%                  c�digos de programas

%--------------------------------------------------------------- CAPÍTULOS %

\chapter{Introdu��o}
\label{cap:intro}

A onipresen�a de software ao redor do globo � indiscut�vel. Consumimos (e produzimos) em uma escala imensur�vel pelos mais otimistas de tempos passados. Temos contato

A qualidade de software aos poucos vem ocupando seu espa�o no processo de desenvolvimento de um software. O testes de software em si, � a principal etapa para que a qualidade seja garantida. Foram realizados alguns trabalhos e pesquisa com base nas diversas t�cnicas de testes, importantes para a realiza��o deste trabalho em si. 

Souza \cite{Souza} realizou uma pesquisa a qual foi a base para alguns crit�rios de testes aplicados neste trabalho. Ela prop�s alguns crit�rios de testes em Banco de Dados Relacional baseados na especifica��o de requisitos atrav�s da UML. Para apoiar a aplica��o dos conjuntos de crit�rios propostos, foi desenvolvido uma abordagem denominada mapeamento Conceitual de Informa��o.


\chapter{Teste de Software}
\label{cap:teste}

%% - - - - - - - - - - - - - - - - - - - - - - - - - - - - - - - - - - -
\section{Conceitos de Teste de Software}
\label{sec:conceitosTesteSoftware}


As op��es da classe s�o \verb|[tese]| (para tese de doutorado), \verb|[dissertacao]| (para disserta��o de mestrado), \verb|[monografia]| (para monografia de curso de especializa��o e \verb|[relatorio]| (para relat�rio final de curso de gradua��o). Se nenhuma op��o for declarada, o documento � considerado como uma disserta��o de mestrado. Se a op��o \verb|[abnt]| for utilizada, as cita��es bibliogr�ficas ser�o geradas conforme definido pelo grupo de trabalho \textsf{abnt-tex}. Contudo, o mais recomend�vel � n�o utilizar essa op��o. Com a op��o \verb|[nocolorlinks]| todos os {\em links} de navega��o no texto ficam na cor preta. O ideal � usar esta op��o para gerar o arquivo para impress�o, pois a qualidade da impress�o dos {\em links} fica superior.


%% - - - - - - - - - - - - - - - - - - - - - - - - - - - - - - - - - - -
\section{T�cnicas e Crit�rios de Teste de Software}
\label{sec:tecnicasCriteriosTesteSoftware}


%% - - - - - - - - - - - - - - - - - - - - - - - - - - - - - - - - - - -
\section{Teste Funcional}
\label{sec:testeFuncional}
Os elementos pr�--textuais s�o definidos p�gina por p�gina, conforme descritos a seguir:


%% lista!
\begin{itemize}
\item O primeiro argumento � o Perfil do aluno; e
\item O segundo argumento � a lista das palavras--chaves para a Ficha Catalogr�fica.
\end{itemize}


%% - - - - - - - - - - - - - - - - - - - - - - - - - - - - - - - - - - -
\section{Teste de Muta��o}
\label{sec:testeMutacao}

X-Man


%% - - - - - - - - - - - - - - - - - - - - - - - - - - - - - - - - - - -
\section{Banco de Dados}
\label{sec:bancoDeDados}
\chapter{Banco de Dados}
\label{cap:bancoDeDados}

% - - - - - - - - - - - - - - - - - - - - - - - - - - - - - - - - - - -
\section{Restri��es de Banco de Dados}
\label{sec:restricoesBd} 

\subsection{Restri��es de Dom�nio}
\label{restricoesDominio}

\subsection{Restri��es de Integridade Sem�ntica}
\label{restricoesIntegridadeSemantica}

\subsection{Restri��es de Integridade Referencial e de Chave Prim�ria}
\label{restricoesIntegridadeReferencial}

% - - - - - - - - - - - - - - - - - - - - - - - - - - - - - - - - - - -
\section{Aplica��es de Banco de Dados}
\label{sec:aplicacoesBd} 

% - - - - - - - - - - - - - - - - - - - - - - - - - - - - - - - - - - -
\section{Linguagem de Banco de Dados}
\label{sec:linguagemBd} 






\chapter{T�cnicas de Testes aplicadas em Banco de Dados Relacional}
\label{tecnicasTesteBD}

\section{Crit�rios de Teste Funcional em Aplica��es de Banco de Dados}
\label{testeFuncionalBd}

\subsection{Crit�rios  Para Testes Intra - Tabelas }
\label{criteriosIntraTabela}

\subsection{Crit�rios  Para Testes Inter-Tabelas }
\label{criteriosInterTabelas}

\section{T�cnica de Teste Mutante em Banco de Dados}
\label{testeMutanteBd}
\chapter{Estudo de Caso}
\label{estudoDeCaso}


\chapter{Plano de Teste}
\label{planoTeste}

% - - - - - - - - - - - - - - - - - - - - - - - - - - - - - - - - - - -
\section{Plano de Testes Baseado no Teste Funcional em BDR}
\label{sec:planoTestesBDR}

Segundo (Souza, 2008 apud Vilela, 2007\cite{Souza}), os Elementos Requeridos (ERs), s�o requisitos que satisfazem o teste, e eles s�o identificados atrav�s dos crit�rios de testes. 

Os elementos requeridos mostrados nas se��es a seguir s�o baseados nos crit�rios propostos na se��o \ref{sec:testeFuncionalBd}, sobre restri��es e cap�tulo 4, utilizando as t�cnicas de teste funcional, por parti��o, por classes de equival�ncia e an�lise do valor limite.

\subsection{Testes Funcionais Intra - Tabelas}
\label{sec:testesIntraTabela}

Na Tabela \ref{tab:elementoRequiridoIntraTabelaChavePrimariaRestricaoUnicidade} s�o apresentados os elementos requeridos pelo crit�rio Restri��o de Unicidade.  

\begin{table*}[p]
	\caption{Restri��o de Chave Prim�ria : Restri��o de Unicidade}
	\label{tab:elementoRequiridoIntraTabelaChavePrimariaRestricaoUnicidade} 
	\begin{tabular}{|c|c|p{3cm}|p{3cm}|p{3cm}|}
		\hline \textbf{Tabela} & \textbf{Atributo} & \textbf{Classes v�lidas} & \textbf{Classes inv�lidas} & \textbf{Sa�da esperada}\\ 
		\hline empresa 
		 & id 
		 & Duas tuplas n�o podem ter o mesmo valor de chave prim�ria do atributo \textit{id}
		 & Duas tuplas possuem o mesmo valor de chave prim�ria 
		 & A chave prim�ria da tabela empresa � �nica, ent�o o BD respeitou a restri��o de unicidade\\ 
		 
		 \hline empresa 
		 & cnpj
		 & Duas tuplas n�o podem ter o mesmo valor para o atributo \textit{cnpj} 
		 & Duas tuplas possuem o mesmo valor para o atributo \textit{cnpj}
		 & O atributo cnpj da tabela empresa � �nico, ent�o o BD respeitou a restri��o de unicidade\\
		
		 \hline empresa 
		 & razao\_social
		 & Duas tuplas n�o podem ter o mesmo valor para o atributo \textit{razao\_social} 
		 & Duas tuplas possuem o mesmo valor para o atributo \textit{razao\_social}
		 & O atributo razao\_social da tabela empresa � �nico, ent�o o BD respeitou a restri��o de unicidade\\
		 
		 \hline filial 
		 & id 
		 & Duas tuplas n�o podem ter o mesmo valor de chave prim�ria do atributo \textit{id}
		 & Duas tuplas possuem o mesmo valor de chave prim�ria 
		 & A chave prim�ria da tabela filial � �nica, ent�o o BD respeitou a restri��o de unicidade\\ 
		 
		 \hline filial 
		 & cnpj
		 & Duas tuplas n�o podem ter o mesmo valor para o atributo \textit{cnpj} 
		 & Duas tuplas possuem o mesmo valor para o atributo \textit{cnpj}
		 & O atributo cnpj da tabela filial � �nico, ent�o o BD respeitou a restri��o de unicidade\\
		 
		 \hline conta 
		 & id 
		 & Duas tuplas n�o podem ter o mesmo valor de chave prim�ria do atributo \textit{id}
		 & Duas tuplas possuem o mesmo valor de chave prim�ria 
		 & A chave prim�ria da tabela conta � �nica, ent�o o BD respeitou a restri��o de unicidade\\ 
		 
		 \hline conta 
		 & cnpj
		 & Duas tuplas n�o podem ter o mesmo valor para o atributo \textit{cnpj} 
		 & Duas tuplas possuem o mesmo valor para o atributo \textit{cnpj}
		 & O atributo cnpj da tabela conta � �nico, ent�o o BD respeitou a restri��o de unicidade\\
		 
		 \hline conta 
		 & razao\_social
		 & Duas tuplas n�o podem ter o mesmo valor para o atributo \textit{razao\_social} 
		 & Duas tuplas possuem o mesmo valor para o atributo \textit{razao\_social}
		 & O atributo razao\_social da tabela conta � �nico, ent�o o BD respeitou a restri��o de unicidade\\
		 
		 \hline plano 
		 & id 
		 & Duas tuplas n�o podem ter o mesmo valor de chave prim�ria do atributo \textit{id}
		 & Duas tuplas possuem o mesmo valor de chave prim�ria 
		 & A chave prim�ria da tabela plano � �nica, ent�o o BD respeitou a restri��o de unicidade\\ 
		 
		 \hline plano 
		 & nome
		 & Duas tuplas n�o podem ter o mesmo valor para o atributo \textit{nome} 
		 & Duas tuplas possuem o mesmo valor para o atributo \textit{nome}
		 & O atributo nome da tabela plano � �nico, ent�o o BD respeitou a restri��o de unicidade\\
		 
		 \hline conta\_plano 
		 & id 
		 & Duas tuplas n�o podem ter o mesmo valor de chave prim�ria do atributo \textit{id}
		 & Duas tuplas possuem o mesmo valor de chave prim�ria 
		 & A chave prim�ria da tabela conta\_plano � �nica, ent�o o BD respeitou a restri��o de unicidade\\ 
		 
		 \hline conta\_plano 
		 & id\_conta, id\_plano
		 & Duas tuplas n�o podem ter o mesmo valor para os atributos \textit{id\_conta, id\_plano} 
		 & Duas tuplas possuem o mesmo valor para os atributos \textit{id\_conta, id\_plano}
		 & Os atributos id\_conta, id\_plano da tabela conta\_plano s�o �nicos, ent�o o BD respeitou a restri��o de unicidade\\
		 
		\hline 
	\end{tabular} 
\end{table*}

Na Tabela \ref{tab:elementoRequiridoIntraTabelaChavePrimariaRestricaoEntidade} s�o apresentados os elementos requeridos pelo crit�rio Restri��o de Entidade. Nesta tabela s�o mostradas as chaves prim�rias de cada Entidade usada no exemplo da Figura \ref{fig:figura3}.

\begin{table*}[hp]
	\caption{Restri��o de Chave Prim�ria : Restri��o de Entidade}
	\label{tab:elementoRequiridoIntraTabelaChavePrimariaRestricaoEntidade} 
	\begin{tabular}{|c|c|p{3cm}|p{3cm}|p{3cm}|}
		\hline \textbf{Tabela} & \textbf{Atributo} & \textbf{Classes v�lidas} & \textbf{Classes inv�lidas} & \textbf{Sa�da esperada}\\ 
		\hline empresa 
		& id 
		& N�o pode conter valores nulos na chave prim�ria do atributo \textit{id}
		& A chave prim�ria \textit{id} cont�m valor nulo.
		& A chave prim�ria da tabela \textit{empresa} n�o possui valor nulo, ent�o o BD respeitou a restri��o de unicidade\\ 
		
		\hline filial 
		& id 
		& N�o pode conter valores nulos na chave prim�ria do atributo \textit{id}
		& A chave prim�ria \textit{id} cont�m valor nulo.
		& A chave prim�ria da tabela \textit{filial} n�o possui valor nulo, ent�o o BD respeitou a restri��o de unicidade\\ 
		
		
		\hline conta 
		& id 
		& N�o pode conter valores nulos na chave prim�ria do atributo \textit{id}
		& A chave prim�ria \textit{id} cont�m valor nulo.
		& A chave prim�ria da tabela \textit{conta} n�o possui valor nulo, ent�o o BD respeitou a restri��o de unicidade\\  
		
		\hline plano 
		& id
		& N�o pode conter valores nulos na chave prim�ria do atributo \textit{id}
		& A chave prim�ria \textit{id} cont�m valor nulo.
		& A chave prim�ria da tabela \textit{plano} n�o possui valor nulo, ent�o o BD respeitou a restri��o de unicidade\\ 
		
		\hline conta\_plano
		& id
		& N�o pode conter valores nulos na chave prim�ria do atributo \textit{id}
		& A chave prim�ria \textit{id} cont�m valor nulo.
		& A chave prim�ria da tabela \textit{conta\_plano} n�o possui valor nulo, ent�o o BD respeitou a restri��o de unicidade\\ 
		
		\hline preco\_modulo
		& id
		& N�o pode conter valores nulos na chave prim�ria do atributo \textit{id}
		& A chave prim�ria \textit{id} cont�m valor nulo.
		& A chave prim�ria da tabela \textit{preco\_modulo} n�o possui valor nulo, ent�o o BD respeitou a restri��o de unicidade\\ 
		
		\hline matriz\_empresa\_modulo
		& id
		& N�o pode conter valores nulos na chave prim�ria do atributo \textit{id}
		& A chave prim�ria \textit{id} cont�m valor nulo.
		& A chave prim�ria da tabela \textit{matriz\_empresa\_modulo} n�o possui valor nulo, ent�o o BD respeitou a restri��o de unicidade\\ 
		
		\hline usuario
		& id
		& N�o pode conter valores nulos na chave prim�ria do atributo \textit{id}
		& A chave prim�ria \textit{id} cont�m valor nulo.
		& A chave prim�ria da tabela \textit{usuario} n�o possui valor nulo, ent�o o BD respeitou a restri��o de unicidade\\ 
		
		\hline grupo
		& id
		& N�o pode conter valores nulos na chave prim�ria do atributo \textit{id}
		& A chave prim�ria \textit{id} cont�m valor nulo.
		& A chave prim�ria da tabela \textit{grupo} n�o possui valor nulo, ent�o o BD respeitou a restri��o de unicidade\\ 
	
		\hline 
	\end{tabular} 
\end{table*}


Nas Figuras \ref{fig:restricaoDominioEmpresa} a \ref{fig:restricaoDominioGrupoPermissao} s�o apresentados os elementos requeridos pelo Dom�nio de Atributos das tabelas definidas no estudo de caso da Figura \ref{fig:figura3}

 
 %% figura 1
 \begin{figure}[H]
 	\centering
 	\includegraphics[width=0.9\textwidth]{./fig/caso-teste/dominio/empresa}
 	\caption{Restri��o de Dom�nio de Atributo da tabela \textit{empresa}}
 	\label{fig:restricaoDominioEmpresa}
 \end{figure}

%% figura 1
\begin{figure}[H]
	\centering
	\includegraphics[width=0.9\textwidth]{./fig/caso-teste/dominio/filial}
	\caption{Restri��o de Dom�nio de Atributo da tabela \textit{filial}}
	\label{fig:restricaoDominioFilial}
\end{figure}

%% figura 1
\begin{figure}[H]
	\centering
	\includegraphics[width=0.9\textwidth]{./fig/caso-teste/dominio/conta}
	\caption{Restri��o de Dom�nio de Atributo da tabela \textit{conta}}
	\label{fig:restricaoDominioConta}
\end{figure}

%% figura 1
\begin{figure}[H]
	\centering
	\includegraphics[width=0.9\textwidth]{./fig/caso-teste/dominio/plano}
	\caption{Restri��o de Dom�nio de Atributo da tabela \textit{plano}}
	\label{fig:restricaoDominioPlano}
\end{figure}

%% figura 1
\begin{figure}[H]
	\centering
	\includegraphics[width=0.9\textwidth]{./fig/caso-teste/dominio/conta-plano}
	\caption{Restri��o de Dom�nio de Atributo da tabela \textit{conta-plano}}
	\label{fig:restricaoDominioContaPlano}
\end{figure}

%% figura 1
\begin{figure}[H]
	\centering
	\includegraphics[width=0.9\textwidth]{./fig/caso-teste/dominio/preco-modulo}
	\caption{Restri��o de Dom�nio de Atributo da tabela \textit{preco-modulo}}
	\label{fig:restricaoDominioPrecoModulo}
\end{figure}

%% figura 1
\begin{figure}[H]
	\centering
	\includegraphics[width=0.9\textwidth]{./fig/caso-teste/dominio/matriz-empresa-modulo}
	\caption{Restri��o de Dom�nio de Atributo da tabela \textit{matriz-empresa-modulo}}
	\label{fig:restricaoDominioMatrizEmpresaModulo}
\end{figure}

%% figura 1
\begin{figure}[H]
	\centering
	\includegraphics[width=0.9\textwidth]{./fig/caso-teste/dominio/usuario}
	\caption{Restri��o de Dom�nio de Atributo da tabela \textit{usuario}}
	\label{fig:restricaoDominioUsuario}
\end{figure}

%% figura 1
\begin{figure}[H]
	\centering
	\includegraphics[width=0.9\textwidth]{./fig/caso-teste/dominio/grupo}
	\caption{Restri��o de Dom�nio de Atributo da tabela \textit{grupo}}
	\label{fig:restricaoDominioGrupo}
\end{figure}

%% figura 1
\begin{figure}[H]
	\centering
	\includegraphics[width=0.9\textwidth]{./fig/caso-teste/dominio/grupo-permissao}
	\caption{Restri��o de Dom�nio de Atributo da tabela \textit{grupo-permissao}}
	\label{fig:restricaoDominioGrupoPermissao}
\end{figure}

%% figura 1
\begin{figure}[H]
	\centering
	\includegraphics[width=0.9\textwidth]{./fig/caso-teste/dominio/conta}
	\caption{Restri��o de Dom�nio de Atributo da tabela \textit{conta}}
	\label{fig:restricaoDominioConta}
\end{figure}

\subsection{Testes Funcionais Inter - Tabelas}
\label{sec:testesInterTabela}

Na Tabela de  \ref{fig:restricaoIntegridadeReferencialEmpresaFilialMatrizEmpresaModulo} a \ref{fig:restricaoIntegridadeReferencialUsuarioGrupoUsuarioGrupoGrupoPermissao} s�o apresentados os elementos requeridos pelo crit�rio Integridade Referencial de Relacionamentos, como especificado na Figura \ref{fig:figura3};

%% figura 1
\begin{figure}[H]
	\centering
	\includegraphics[width=0.9\textwidth]{./fig/caso-teste/integridade-referencial/empresa-filial-matrizEmpresaModulo}
	\caption{Restri��o de Integridade Referencial entre as tabelas \textit{empresa}, \textit{filial} e \textit{matriz\_empresa\_modulo}}
	\label{fig:restricaoIntegridadeReferencialEmpresaFilialMatrizEmpresaModulo}
\end{figure}

\begin{figure}[H]
	\centering
	\includegraphics[width=0.9\textwidth]{./fig/caso-teste/integridade-referencial/empresa-contaPlano-conta}
	\caption{Restri��o de Integridade Referencial entre as tabelas \textit{empresa}, \textit{conta\_plano} e \textit{conta}.}
	\label{fig:restricaoIntegridadeReferencialEmpresaContaPlanoConta}
\end{figure}


%% figura 1
\begin{figure}[H]
	\centering
	\includegraphics[width=0.9\textwidth]{./fig/caso-teste/integridade-referencial/contaPlano-conta-precoModulo}
	\caption{Restri��o de Integridade Referencial entre as tabelas \textit{conta\_plano}, \textit{plano} e \textit{preco\_modulo}}
	\label{fig:restricaoIntegridadeReferencialContaPlanoContaPrecoModulo}
\end{figure}


%% figura 1
\begin{figure}[H]
	\centering
	\includegraphics[width=0.9\textwidth]{./fig/caso-teste/integridade-referencial/usuario-grupo-usuarioGrupo-grupoPermissao}
	\caption{Restri��o de Integridade Referencial entre as tabelas \textit{usuario}, \textit{grupo}, \textit{usuario\_grupo} e \textit{grupo\_permisso}.}
	\label{fig:restricaoIntegridadeReferencialUsuarioGrupoUsuarioGrupoGrupoPermissao}
\end{figure}



%------------------------------------------------------------ BIBLIOGRAFIA %
\cleardoublepage
\nocite{*} %%% Retire esta linha para gerar a bibliografia com apenas as
           %%% refer�ncias usadas no seu texto!
\arial
\bibliography{./bib/modelo-tese} %%% Nomes dos seus arquivos .bib
\label{ref-bib}

%--------------------------------------------------------------- AP�NDICES %
\apendices

\chapter{Script de cria��o da base de dados}
\label{apend:A}
Ap�ndicess s�o iniciados com o comando \verb|\apendices|.
\chapter{Script de Inser��o inicial no Banco}
\label{apend:B}
Texto do Ap�ndice~\ref{apend:B}.

Ap�ndices s�o iniciados com o comando \verb|\apendices|.

\end{document}

%------------------------------------------------------------------------- %
%        F I M   D O  A R Q U I V O :  m o d e l o - t e s e . t e x       %
%------------------------------------------------------------------------- %
